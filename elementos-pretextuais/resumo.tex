Em tempos onde diversos dados governamentais passam a ser disponibilizados de maneira transparente para o público, começam a surgir diversas soluções de terceiros que fazem uso desses dados para auxiliar a população. Esses, agora processados e analisados, tornam-se ferramentas de informação nas mãos do cidadão e promovem a fiscalização dos serviços públicos prestados.

O escopo deste trabalho é baseado nos dados abertos da frota de ônibus da cidade do Rio de Janeiro, um dado fornecido pela Prefeitura carioca e que diz respeito às linhas municipais e à geolocalização de todos os ônibus em tempo real. Utilizaremos esses dados para elaborar algoritmos que permitem analisar o real funcionamento dos ônibus do Rio. Tais algoritmos colhem estatísticas que podem servir como um documento para a fiscalização desse serviço público tão essencial.
