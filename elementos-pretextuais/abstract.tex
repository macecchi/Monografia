In times where all kinds of governamental data become available in a transparent manner to the public, different third-party solutions arise making use of this data to help people. That data, now processed and analyzed, become information tools on the citizen's hands and promote the fiscalization of public services.

The studies presented in this document are based on the open data of Rio de Janeiro's city buses, made available by Rio's city hall, composed of information about the municipal bus lines and the geolocalization of all of their buses in real-time. We'll gather data to elaborate algorithms that allow the analysis of the real functioning of Rio's buses. Those algorithms gather statistics that can be used to promote the accountability of such an essential public service.